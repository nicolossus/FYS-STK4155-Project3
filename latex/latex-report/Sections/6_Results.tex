%================================================================
\section{Results and Discussion}\label{sec:Results}
%================================================================
\begin{table}[H]
\caption{Table text}
\centering
\rowcolors{2}{gray!20}{white}
\begin{tabular}{ccc}
\hline
\hline 
$\alpha$ & $\beta$ & $\gamma$
\\
\hline 
\hline 
0.1 & 0.2 & 0.3
\\
0.4 & 0.5 & 0.6
\\
0.7 & 0.8 & 0.9
\\
\hline
\hline 
\end{tabular}
\label{tab:tab1}
\end{table}


%===============================================================
\subsection{Heat Equation}\label{sec:heateq results}
%===============================================================
Accompanying notebook: \href{https://github.com/nicolossus/FYS-STK4155-Project3/blob/master/notebooks/heat_pde_with_fe_and_tf.ipynb}{heat\_pde\_with\_fe\_and\_tf.ipynb}.

%===============================================================
\subsubsection{Forward Euler}
%===============================================================
Nx = 11
Nt = 199

Max diff: 0.04009472326320718
Mean diff: 0.003186288853514369

\autoref{fig:heat_fe}
\begin{figure}[H]
\centering
\subfloat[]{{\includegraphics[scale=0.48]{latex/figures/heat_ana_fe.pdf}}}
\qquad
\subfloat[]{{\includegraphics[scale=0.48]{latex/figures/heat_fe.pdf}}}
\qquad
\subfloat[]{{\includegraphics[scale=0.5]{latex/figures/heat_diff_fe.pdf}}}
\caption{FE heat}
\label{fig:heat_fe}
\end{figure}

CPU benchmark
Number of iterations: 1000
FE mean CPU time: 0.00158 secs

%===============================================================
\subsubsection{FFNN Model 1}
%===============================================================

Model 1: Train on spatial and temporal points as dictated by FE stability criterion

Nx = 11
Nt = 199
2 layers + output, [150, 50, 1], [tanh, sigmoid, none] 
1000 epochs
Adam, initial lr=0.01
Step: 1000, Loss: 0.008064055285575904
Training FFNN CPU time: 147.62562 secs

\autoref{fig:heat_nn1} Plot solution on the grid FFNN is trained on

Max diff: 0.008466789818225778
Mean diff: 0.0029635112954797672

\begin{figure}[H]
\centering
\subfloat[]{{\includegraphics[scale=0.48]{latex/figures/heat_ana_nn1.pdf}}}
\qquad
\subfloat[]{{\includegraphics[scale=0.48]{latex/figures/heat_nn1.pdf}}}
\qquad
\subfloat[]{{\includegraphics[scale=0.5]{latex/figures/heat_diff_nn1.pdf}}}
\caption{Model 1, Plot solution on the grid FFNN is trained on}
\label{fig:heat_nn1}
\end{figure}

\autoref{fig:heat_nn2} Plot solution on a larger grid than FFNN is trained on, i.e., points must be interpolated

301, 301 points

Max diff: 0.008467176325477247
Mean diff: 0.0032787069037113277

\begin{figure}[H]
\centering
\subfloat[]{{\includegraphics[scale=0.48]{latex/figures/heat_ana_nn2.pdf}}}
\qquad
\subfloat[]{{\includegraphics[scale=0.48]{latex/figures/heat_nn2.pdf}}}
\qquad
\subfloat[]{{\includegraphics[scale=0.5]{latex/figures/heat_diff_nn2.pdf}}}
\caption{Model 1, Plot solution on a larger grid than FFNN is trained on, i.e., points must be interpolated}
\label{fig:heat_nn2}
\end{figure}

%===============================================================
\subsubsection{FFNN Model 2}
%===============================================================
Model 2: Train on equal number of spatial and temporal points

Nx = 11
Nt = 11
2 layers + output, [150, 50, 1], [tanh, sigmoid, none] 
1000 epochs
Adam, initial lr=0.01
Step: 1000, Loss: 0.008064055285575904
Training FFNN CPU time: 147.62562 secs
Step: 1000, Loss: 0.0027709536906761877
Training FFNN CPU time: 42.11612 secs

\autoref{fig:heat_nn3} Plot solution on the grid FFNN is trained on

Max diff: 0.026960711104318635
Mean diff: 0.0029851059330631173

\begin{figure}[H]
\centering
\subfloat[]{{\includegraphics[scale=0.48]{latex/figures/heat_ana_nn3.pdf}}}
\qquad
\subfloat[]{{\includegraphics[scale=0.48]{latex/figures/heat_nn3.pdf}}}
\qquad
\subfloat[]{{\includegraphics[scale=0.5]{latex/figures/heat_diff_nn3.pdf}}}
\caption{Model 2, Plot solution on the grid FFNN is trained on}
\label{fig:heat_nn3}
\end{figure}

\autoref{fig:heat_nn4} Plot solution on a larger grid than FFNN is trained on, i.e., points must be interpolated

301, 301 points

Max diff: 0.03215260678986426
Mean diff: 0.003950743671390742

\begin{figure}[H]
\centering
\subfloat[]{{\includegraphics[scale=0.48]{latex/figures/heat_ana_nn4.pdf}}}
\qquad
\subfloat[]{{\includegraphics[scale=0.48]{latex/figures/heat_nn4.pdf}}}
\qquad
\subfloat[]{{\includegraphics[scale=0.5]{latex/figures/heat_diff_nn4.pdf}}}
\caption{Model 2, Plot solution on a larger grid than FFNN is trained on, i.e., points must be interpolated}
\label{fig:heat_nn4}
\end{figure}

%===============================================================
\subsubsection{Home-made FFNN}
%===============================================================


%===============================================================
%===============================================================
\subsection{Eigenvalue Problem}\label{sec:eigenvalue results}
%===============================================================
%===============================================================

Accompanying notebook: \href{https://github.com/nicolossus/FYS-STK4155-Project3/blob/master/notebooks/eigenvalue_tf.ipynb}{eigenvalue\_tf.ipynb}.

%===============================================================
\subsubsection{Benchmark Problem 1}
%===============================================================

\autoref{tab:parabench1} tabulates the problem and model parameters for Benchmark Problem 1 described in \autoref{sec:benchmark problem 1}. 

\begin{table}[H]
\caption{Problem and model parameters for Benchmark Problem 1.}
\centering
\rowcolors{2}{gray!20}{white}
\begin{tabular}{c|c}
\hline
\hline 
Parameter & Numerical Value
\\
\hline 
\hline 
Initial vector & $\bm{x_0}=(1,0,0)$
\\
Simulation time & 1
\\
Number of time points (Euler) & 101
\\
Number of time points (FFNN) & 11
\\
Number of epochs (FFNN) & 2000
\\
\hline
\hline 
\end{tabular}
\label{tab:parabench1}
\end{table}


\autoref{fig:benchrun1} shows the results for Benchmark Problem 1. The computed Rayleigh quotients and the absolute error relative to the eigenvalue computed by Numpy are tabulated in \autoref{tab:eigbench1}. 

\begin{figure}[H]
\centering
\subfloat[]{{\includegraphics[scale=0.6]{latex/figures/eigvec_comp_benchrun1.pdf}}}
\qquad
\subfloat[]{{\includegraphics[scale=0.6]{latex/figures/eigval_benchrun1.pdf}}}
\caption{Results for Benchmark Problem 1 with a $3\times 3$ real symmetric matrix, $A$. \textbf{(a)} shows the components of the computed steady-state vector as a function of time. The dashed lines are the components computed by Euler's method and the solid lines are computed by the FFNN model. The dotted lines are the normalized (unit "length") eigenvector components corresponding to the largest eigenvalue computed directly from the matrix by Numpy's linalg.eig. \textbf{(b)} shows the computed Rayleigh quotients, $r$, as a function of time for both Euler's method and the FFNN model and the largest eigenvalue, $\lambda_\mathrm{max}$, of the matrix computed by Numpy's linalg.eig as well. The final Rayleigh quotients are rounded to 5 decimal points.}
\label{fig:benchrun1}
\end{figure}

\begin{table}[H]
\caption{The computed Rayleigh quotients at the final simulation time for both Euler's method and the FFNN model. The absolute error relative to the eigenvalue computed by Numpy is also listed.}
\centering
\rowcolors{2}{gray!20}{white}
\begin{tabular}{c|c|c}
\hline
\hline 
Method & Rayleigh Quotient & Absolute Error
\\
\hline 
\hline 
Numpy & 8.0 & –
\\
Euler & 7.99999993 & $6.007 \cdot 10^{-8}$  
\\
FFNN & 7.99999803 & $1.967 \cdot 10^{-6}$
\\
\hline
\hline 
\end{tabular}
\label{tab:eigbench1}
\end{table}

The eigenvalue computed by Numpy correspond to the analytical solution in this case. Euler achieves higher accuracy

DISCUSSION

%===============================================================
\subsubsection{Benchmark Problem 2}
%===============================================================

The problem and model parameters for Benchmark Problem 2, described in \autoref{sec:benchmark problem 2}, are the same as for Benchmark Problem 1 tabulated in \autoref{tab:parabench1}. The difference for this problem is that the matrix has opposite signs, $-A$, i.e. we are finding the smallest eigenvalue of $A$.

\autoref{fig:benchrun2} shows the results for Benchmark Problem 1. The computed Rayleigh quotients and the absolute error relative to the eigenvalue computed by Numpy are tabulated in \autoref{tab:eigbench2}. 

\begin{figure}[H]
\centering
\subfloat[]{{\includegraphics[scale=0.6]{latex/figures/eigvec_comp_benchrun2.pdf}}}
\qquad
\subfloat[]{{\includegraphics[scale=0.6]{latex/figures/eigval_benchrun2.pdf}}}
\caption{Results for Benchmark Problem 2 with a $3\times 3$ real symmetric matrix, $-A$. \textbf{(a)} shows the components of the computed steady-state vector as a function of time. The dashed lines are the components computed by Euler's method and the solid lines are computed by the FFNN model. The dotted lines are the normalized (unit "length") eigenvector components corresponding to the largest eigenvalue computed directly from the matrix by Numpy's linalg.eig. \textbf{(b)} shows the computed Rayleigh quotients, $r$, as a function of time for both Euler's method and the FFNN model and the largest eigenvalue, $\lambda_\mathrm{max}$, of the matrix computed by Numpy's linalg.eig as well. The final Rayleigh quotients are rounded to 5 decimal points.}
\label{fig:benchrun2}
\end{figure}

\begin{table}[H]
\caption{The computed Rayleigh quotients at the final simulation time for both Euler's method and the FFNN model. The absolute error relative to the eigenvalue computed by Numpy is also listed.}
\centering
\rowcolors{2}{gray!20}{white}
\begin{tabular}{c|c|c}
\hline
\hline 
Method & Rayleigh Quotient & Absolute Error
\\
\hline 
\hline 
Numpy & 1.0 & –
\\
Euler & 0.99999995 & $4.112 \cdot 10^{-8}$  
\\
FFNN & 0.99997506 & $2.493 \cdot 10^{-5}$
\\
\hline
\hline 
\end{tabular}
\label{tab:eigbench2}
\end{table}


DISCUSS


%===============================================================
\subsubsection{Benchmark Problem 3}
%===============================================================

