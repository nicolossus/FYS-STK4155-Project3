%================================================================
\section{Introduction}\label{sec:Introduction}
%================================================================

Differential equations have a wide range of applications in the natural sciences. The aim of this project is to design feed-forward neural network (FFNN) models suitable for solving differential equations. 

extracting eigenvalues




In this notebook, an FFNN model is employed within the TensorFlow framework to learn the solution to the nonlinear, coupled ODE, presented by Yi et. al in the article from [Computers and Mathematics with Applications 47, 1155 (2004)](https://www.sciencedirect.com/science/article/pii/S0898122104901101), describing the state of a CTRNN model. Given a real symmetric matrix $A$ in the source term, the temporal dynamic described by this ODE has convergence properties to an eigenvector corresponding to the largest eigenvalue $\lambda$, given that the initial vector $\mathbf{x}_0$ is not orthogonal to the eigenspace of $\lambda$. If $\mathbf{x}_0$ is not orthogonal to the eigenspace of the smallest eigenvalue $\sigma$, replacing $A$ with $-A$ yields an eigenvector corresponding to $\sigma$. The eigenvalue itself is calculated with the Rayleigh quotient.

The aim is to check if the constructed FFNN succeeds in computing both the largest and smallest eigenvalue for some benchmark $3\times 3$ and $6\times 6$ real symmetric matrices. We will also try choosing $\mathbf{x}_0$ orthogonal to the eigenspace corresponding to the largest eigenvalue, to make the network converge to an eigenvalue different from the largest. This starting point will itself be an eigenvector.

In order to assess the FFNN model, we will compare the result with those from Euler's method for solving the same ODE and Numpy's ``linalg.eig`` which directly computes the eigenvalues of the matrix $A$. 

Unfortunately, we found tensorflow objects (such as tensorflow functions) hard to reuse. There is therefore much repetitive code in the code cells below. This notebook is organized such that each code cell have all the necessary code, except imports, to solve a particular problem. To better distinguish what problem a particular code cell solves, we have added headers that explains the problem at hand.

%%%
%%%
%%%
%%%

and computes the corresponding eigenvector upon time evolution of the model. 

The temporal dynamic of the CTRNN describes the state of the network, and has convergence properties to the steady  

given a real symmetric matrix, solution gives a steady-state vector which can be taken as an

The proposed network model is described by differential equations, which is a class of continuous time recurrent neural network model. 

A feed-forward neural network is presented for computing the largest and smallest eigenvalue of a real symmetric matrix, A.


Efficient computation of eigenvectors and eigenvalues of a matrix is an important problem in engineering, especially for computing eigenvectors corresponding to largest or smallest eigenvalues of a matrix

A FFNN model was employed to learn the initial-boundary problem given by the heat equation in one spatial dimension. The results were compared with those from a closed form solution and an explicit numerical scheme using the Forward Euler (FE) method.

in this project we will employ a feed forward neural network (ffnn) to learn the initial-boundary problem given by the heat equation, a partial differential equation (pde) with both temporal and spatial derivatives, and compare the results with those from an explicit scheme using the forward euler (fe) method. the problem formulation will be in one spatial dimension with initial and boundary conditions that yield a closed-form solution. the analytical solution will be used to assess the accuracy of the methods. 

to further study the capability of a ffnn to learn the solution of differential equations, we will 
temporal dynamic of neural network. given a real symmetric matrix, solution gives a steady-state vector which can be taken as an eigenvector of the matrix. by computing the rayleigh quotient, largest eigenvalue.
compare with euler's method and traditional numerical eigenvalue solvers.


Employ a feed-forward neural network to learn the solution of the heat equation in one spatial dimension. The network was trained on a grid with 11 points with the both the spatial, $x$, and temporal, $t$, domain $x, t \in [0,1]$. The trained network was then given a larger grid with 41 points to interpolate the solution on. Mean difference from analytic solution ...


