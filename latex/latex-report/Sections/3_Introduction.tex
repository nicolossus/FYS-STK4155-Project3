%================================================================
\section{Introduction}\label{sec:Introduction}
%================================================================

Nyttige lenker:

\url{https://krisbhei.github.io/DEnet/DNN_Diffeq/DNN_diffeq_html.html?fbclid=IwAR2XiOCNNf8XFbj3xtk3JeVaIkvJ4bDKAo2uo_fa7nwsaCli0I-CjNfI2vw}

\url{https://compphysics.github.io/MachineLearning/doc/pub/odenn/html/odenn.html?fbclid=IwAR2otu_WD_5hCfc7qYPVEGU5oso4qxB56KVc9xYerbxj1_qZnwzN7oB2MWU}

\url{https://becominghuman.ai/neural-networks-for-solving-differential-equations-fa230ac5e04c} 

in this project we will employ a feed forward neural network (ffnn) to learn the initial-boundary problem given by the heat equation, a partial differential equation (pde) with both temporal and spatial derivatives, and compare the results with those from an explicit scheme using the forward euler (fe) method. the problem formulation will be in one spatial dimension with initial and boundary conditions that yield a closed-form solution. the analytical solution will be used to assess the accuracy of the methods. 

to further study the capability of a ffnn to learn the solution of differential equations, we will 
temporal dynamic of neural network. given a real symmetric matrix, solution gives a steady-state vector which can be taken as an eigenvector of the matrix. by computing the rayleigh quotient, largest eigenvalue.
compare with euler's method and traditional numerical eigenvalue solvers.


Theory

The Heat Equation 

Closed Form of the Heat Equation

Explicit Numerical Scheme Using Forward Euler

Feed Forward Neural Network

Solving the Heat Equation with FFNN 
- trial solution

Eigenvalue Problem
- stuff from paper
- Euler's method for solving ODE (for comparison)
- Numpy eig
