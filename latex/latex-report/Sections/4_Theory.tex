%================================================================
\section{Theory}\label{sec:Theory}
%================================================================




%----------------------------------------------------------------
\subsection{Project Theory 1}\label{sec:project theory}
%----------------------------------------------------------------
This is \autoref{sec:project theory}.

\begin{equation}\label{eq:heat}
\frac{\partial^2 u(x,t)}{\partial x}= \frac{\partial u(x,t)}{\partial t},\;\;\;\;t>0,x\in[0,1],
\end{equation}
with initial conditions
\begin{equation*}
    u(x,0)=\sin(\pi x),\;\;\;\;0<x<1
\end{equation*}
and boundary conditions
\begin{equation*}
    u(0,t)=0,\;\;\;\;t\ge0
\end{equation*}
and
\begin{equation*}
    u(1,t)=0,\;\;\;\;t\ge0.
\end{equation*}

Assume that the solution has the form
\begin{equation*}
    u(x,t)=f(x)g(t).
\end{equation*}
By the initial conditions, we get that
\begin{equation*}
    f(x)=k^{-1}\sin(\pi x),
\end{equation*}
where $k=g(0)$. Then, from \cref{eq:heat},
\begin{equation*}
    f''(x)g(t) = f(x)g'(t)\;\;\;\Leftrightarrow\;\;\;\frac{f''(x)}{f(x)}=\frac{g'(t)}{g(t)}.
\end{equation*}
Plugging in for $f(x)$ and $f''(x)=-\pi^2a^{-1}\sin(\pi x)$, we find that
\begin{equation*}
    -\pi^2 = \frac{g'(t)}{g(t)}.
\end{equation*}
We integrate both sides with respect to $t$ to obtain
\begin{equation*}
    -\pi^2t+C = \int\frac{dg}{dt}\frac{1}{g}dt = \int\frac{1}{g}dg =\log |g|.
\end{equation*}
Hence
\begin{equation*}
    g(t)=D\mathrm{e}^{-\pi^2 t},
\end{equation*}
where $D$ is some constant. Since $g(0)=D$, we have $D=k$. We get that
\begin{equation*}
    u(x,t)=g(t)f(x)=\mathrm{e}^{-\pi^2 t}\sin(\pi x),
\end{equation*}
which can be checked to be a solution of \cref{eq:heat} satisfying the initial and boundary conditions. Next we show that this solution is unique.

Suppose that that $u_1$ and $u_2$ are solutions. Define $v(x,t)=u_1(x,t)-u_2(x,t)$ and let
\begin{equation*}
    w(t)=\frac{1}{2}\int_0^1v(x,t)^2\,\dd x,\;\;\;\; t\ge0.
\end{equation*}
Note that by \cref{eq:heat},
\begin{equation*}
    \frac{\partial^2 u(x,t)}{\partial x}= \frac{\partial u(x,t)}{\partial t},\;\;\;\;t>0,x\in[0,1],
\end{equation*}
by the initial condition $w(0)=0$, and by the boundary condition $v(0,t)=v(1,t)=0$ for $t>0$.

By Leibniz' rule \cite[8.11.2]{die69}, we can differentiate under the integral sign, so that
\begin{equation*}
    w'(t) = \int_0^1v(x,t)\pdv{v(x,t)}{t}\,\dd x = \int_0^1 v(x,t)\pdv[2]{v(x,t)}{x}\,\dd x.
\end{equation*}
Then integration by parts yields
\begin{equation*}
    w'(t) = \eval{v(x,t)\pdv{v(x,t)}{x}}_{x=0}^1 - \int_0^1\qty(\pdv{v(x,t)}{x})^2\,\dd x = -\int_0^1\qty(\pdv{v(x,t)}{x})^2\dd x,
\end{equation*}
implying that $w'(t)\le 0$ for all $t>0$. On the other hand, since $w(t)\ge0$ for all $t>0$ and $w(0)=0$, we must have $w'(t)\ge0$. It follows that $w'(t)=0, t>0$, and thus $w(t)=0$ for all $t\ge0$. We conclude that $u_1=u_2$.



Let $A\in\mathbb{R}^{n\times n}$ be a real symmetric matrix. Define $f\colon\mathbb{R}^n\to\mathbb{R}^n$ by
\[
f(x)=[x^TxA+(1-x^TAx)I]x,\;\;\;\; x\in\mathbb{R}^n,
\]
where $I\in\mathbb{R}^{n\times n}$ is the identity matrix. Let $x\colon\mathbb{R}\to\mathbb{R}^n$ be a map that satisfies
\begin{equation}\label{eq:diff_eigen}
    Dx(t)=-x(t)+f(x(t)).
\end{equation}
We then have the following theorem \cite{yfh04}:
\begin{theorem}
For each solution $x\colon\mathbb{R}\to\mathbb{R}^n$ of \autoref{eq:diff_eigen}, the limit $\lim_{t\to\infty}x(t)$ exists and converges to an eigenvector of $A$.

If $\lambda$ is the largest eigenvalue of $A$ and the starting point $x(0)$ is not orthogonal to the eigenspace of $\lambda$, then $\lim_{t\to\infty}x(t)$ is an eigenvector of $A$ with eigenvalue $\lambda$.

Replacing $A$ with $-A$, then if $x(0)$ is not orthogonal to the eigenspace of the smallest eigenvalue $\sigma$ of $A$, the limit $\lim_{t\to\infty}x(t)$ converges to an eigenvector corresponding to $\sigma$.
\end{theorem}

\begin{definition}[Rayleigh quotient {\cite[234]{hj13}}]
Given a matrix $A\in\mathbb{R}^{n\times n}$ and a vector $x\in\mathbb{R}^n$, the \emph{Rayleigh quotient} is defined as
\begin{equation*}
    r(A,x) = \frac{x^TAx}{x^Tx}.
\end{equation*}
Note that if $x$ is an eigenvector of $A$ with eigenvalue $\lambda$, then
\begin{equation*}
    r(A,x)=\frac{x^TAx}{x^Tx}=\frac{x^T (\lambda x)}{x^T x}=\lambda.
\end{equation*}
\end{definition}